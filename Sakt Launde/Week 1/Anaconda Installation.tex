Anaconda is an open-source software that contains Jupyter, spyder, etc. that are used for large data processing, data analytics, heavy scientific computing and Machine Learning.

To start working we have to install it on our PC. Here is the installation process:

1. We go www.anaconda.com and download the latest version of Anaconda.

2. After the download is complete, we run the setup.

3. License Agreement: Now we have to read and agree to the License Agreement by clicking on "I Agree". 
   It is important that you understand the License Agreement before clicking "I agree".

4. Select Installation Type: Select the installation type as "Just Me".

5. Choose Install Location: Now we select the install location by typing the location address or by browsing. 
   Remember that you need to have atleast 3.5 GB of disk space free at the location. 
   Also, choose a folder without a 'space' character in its name to avoid any problems in the future.

6. Advanced Installation Option: Tick both the boxes "Add Anaconda to my PATH environment variable" and "Register Anaconda as my default Python 3.9". 
   Then click "Install".

7. Installing: Now we patiently wait while the installation process completes.

8. Once the installation is complete, we click on "Finish" and close the setup.

Now with Anaconda installed on our system we just need to open Anaconda Navigator to use the tools Anaconda has to offer.